% !Mode::"UTF-8"
\documentclass[12pt]{article}

% 页面设置
\usepackage{geometry}
\geometry{left=2.5cm, right=2.5cm, top=2.5cm, bottom=2.5cm}  % 设置页边距
\usepackage{graphicx}  % 引用图片
\usepackage{ctex}
\usepackage{fontspec}
\usepackage{setspace}

% 字体设置
\setmainfont{Times New Roman}
\setCJKmainfont{SimSun}
\setCJKsansfont{SimHei}

% 表格设置
\usepackage{makecell}
\newcommand{\addcell}[2][4]{\makecell{\zihao{#1}\textsf{#2}}}
\usepackage{titlesec}
\usepackage{booktabs}
\usepackage{tabularx}

% 设置图注、表注
\usepackage{caption}
\usepackage{bicaption}
\captionsetup{labelsep=quad, font={small, bf}, skip=2pt}
\DeclareCaptionOption{english}[]{
    \renewcommand\figurename{Fig.}
    \renewcommand\tablename{Table}
}
\captionsetup[bi-second]{english}

% 设置页眉
\usepackage{fancyhdr}
\pagestyle{fancy}
\fancypagestyle{preContent}{
    \fancyhead[L]{\zihao{-5} 物理化学实验}
    \fancyhead[C]{\zihao{-5} 实验xx\ \ xxxxx}
    \fancyhead[R]{\zihao{-5} 1800011716\ 王崇斌}
}
\pagestyle{preContent}

%	设置首页页眉页脚
\fancypagestyle{plain}{
	\fancyhead[L]{\zihao{-5} 物理化学实验}
	\fancyhead[C]{\zihao{-5} 实验xx\ \ xxxxx}
	\fancyhead[R]{\zihao{-5} 1800011716\ 王崇斌}
	\cfoot{}
}

% 设置标题格式
\titleformat*{\section}{\zihao{4}\sffamily}
\titleformat*{\subsection}{\zihao{-4}\sffamily}
\titleformat*{\subsubsection}{\zihao{-4}\sffamily}
\titlespacing*{\section}{0pt}{10pt}{10pt}
\titlespacing*{\subsection}{0pt}{10pt}{5pt}
\titlespacing*{\subsubsection}{0pt}{10pt}{5pt}

% 设置引用格式
\usepackage[super,round,comma,compress]{natbib}
\usepackage{hyperref}  % 使用hyperref包,可以提供文献引用到文件末尾


% 一些相关的包
\usepackage{amsmath}  % 数学公式
\usepackage{amssymb}  % 特殊字符
\usepackage[version=4]{mhchem}  % 用于输入化学式
\usepackage{braket}  % 用于输入Dirac符号
\usepackage{subfigure}  % 多张图片的排版

% 定义常用的命令
\def\d{\mathrm{d}}  % 正体的常用数学常数
\def\e{\mathrm{e}}
\def\i{\mathrm{i}}
\def\dps{\displaystyle}  % 
\newcommand{\mr}[1]{\mathrm{#1}}
\newcommand{\mb}[1]{\mathbf{#1}}
\newcommand{\dv}[2]{\frac{\d{#1}}{\d{#2}}}  % 定义导数、偏导数的简便记号
\newcommand{\pdv}[2]{\frac{\partial{#1}}{\partial{#2}}}
\def\degree{$^{\circ}$}  % 角度
\def\celsius{^{\circ}\mr{C}}  % 摄氏度

%正文
\begin{document}
    % 标题页
    \begin{titlepage}
    	% 页眉
    	\thispagestyle{plain}
        % 图片
        \begin{figure}[h]
            \centering
            \includegraphics{pku.png}
        \end{figure}
        \vspace{24pt}
        % 标题
        \centerline{\zihao{-0} \textsf{物理化学实验报告}}
        \vspace{40pt} % 空行
        \begin{center}
            \begin{tabular}{cp{14.1cm}}
                % 题目
                \addcell[2]{题目:\ } & \addcell[2]{xxxxxx} \\
                \cline{2-2}
            \end{tabular}
        \end{center}
        \vspace{20pt} % 空行
        \begin{center}
            \doublespacing
            \begin{tabular}{cp{5cm}}
                % 姓名
                \addcell{姓\phantom{空格}名:\ } & \addcell{王崇斌} \\
                \cline{2-2}
                % 学号
                \addcell{学\phantom{空格}号:\ } & \addcell{1800011716}\\
                \cline{2-2}
                % 组别
                \addcell{组\phantom{空格}别:\ } & \addcell{11组} \\
                \cline{2-2}
                % 实验日期
                \addcell{实验日期:\ } & \addcell{2020.xx.xx}\\
                \cline{2-2}
                % 室温
                \addcell{室\phantom{空格}温:\ } & \addcell{xxx.xx\ K}\\
                \cline{2-2}
                % 大气压强
                \addcell{大气压强:\ } & \addcell{xxx.xx\ kPa}\\
                \cline{2-2}
            \end{tabular}
            \begin{tabular*}{\textwidth}{c}
                \\ % 这是空行
                \\ % 这是空行
                \\ % 这是空行
                \\ % 这是空行
                \hline % 分割线
            \end{tabular*}
        \end{center}
        % 摘要
        \textsf{摘\ \ 要}\ \ 本实验通过...,利用...,得到了...,比较了...,从而初步了解了...。
        \\
        \\
        % 关键字
        \textsf{关键词}\ \ xxx;xxx;xxx;xxx
    \end{titlepage}

    \section{引言}
	引言部分...
               
	\vbox{}        
    \section{实验部分}
    	\subsection{仪器和试剂}
    	仪器和试剂...
    	
    	 \subsection{实验内容}
			\subsubsection{实验内容1}
			实验内容1...
			\subsubsection{实验内容2}
		实验内容2...
			\subsubsection{实验内容3}
			实验内容3...
			\subsubsection{实验内容4}
			实验内容4...
			\subsubsection{实验内容5}
			实验内容5...
    	
	\vbox{}  
	\section{数据与结果}
 		\subsection{实验数据记录及处理}
 		\subsubsection{实验内容1}
 		实验内容1...
 		
 		 \begin{figure}[h]
 			\centering
 			\includegraphics[width=0.5\textwidth]{pku.png}
 			\bicaption{中文图题}{Caption}
 		\end{figure}
 	
 		xxx如\textbf{表1}所示。
 		
 		  \begin{table}[h]
 			\centering
 			\zihao{5}
 			\bicaption{中文表题}{Caption}
 			\begin{tabular}{ccccc}
 				\toprule
 				一 & 二 & 三 & 四 & \thead[c] {$E_{trans}$ \\ / $ \rm kJ \cdot mol^{-1} $} \\
 				\midrule
 				1 & 1 & 1 & 3 & $-1.01316\times10^{6}$ \\
 				2 & 2 & 1 & 7 & \\
 				3 & 3 & 1 & 3 & \\
 				4 & 4 & 1 & 6 & \\
 				5 & 5 & 1 & 1 & \\
 				\bottomrule
 			\end{tabular}
 		\end{table}
 	\vbox{}
 		\subsubsection{实验内容2}
		实验内容2...
		\subsubsection{实验内容3}
		实验内容3...
		\subsubsection{实验内容4}
		实验内容4...
		\subsubsection{实验内容5}
		实验内容5...
 		$$ E_{total}=E_{trans}+E_{rot}+E_{vib}+E_{elec} $$ 
 	
 		$$ 1\ \ {\rm a.u.}=2625.50 \ \ {\rm kJ \cdot mol^{-1}} $$
 		$$\Delta E_{total}=E_{total}({\rm nap})-E_{total}{\rm (azu)=-1.4\times10^{2} \ \ kJ \cdot mol^{-1} }$$
 	
 		参考值\citealp{dean1992lange} 
 		$$\mu=1.66\ \ {\rm D}$$
 		实验值\citealp{Bast2009,Konecny2019}
 		$$E_{\rm Zn}=5.80 \ \ {\rm eV}\ \ \ \ \ \ \ E_{\rm Cu}=1.39\ \ {\rm eV}$$
 		
 		
 		\subsection{实验结果及分析}
 		\subsubsection{实验内容1}
 		实验内容1...
 		
 		

 	
	\vbox{}  	
 	\section{讨论与结论}
		\subsection{实验讨论}
 		\subsubsection{实验讨论1}
 	 	实验讨论1...
 	 	\subsubsection{实验讨论2}
 	 	实验讨论2...
 	 	\subsubsection{实验讨论3}
 		实验讨论3...
 	 
 	 	\subsection{实验结论}
 	 	实验结论...

	\vbox{}  

	\bibliographystyle{achemso}
	\bibliography{cite}



\end{document}