% !Mode::"UTF-8"
\documentclass[12pt]{article}

% 页面设置
\usepackage{geometry}
\geometry{left=2.5cm, right=2.5cm, top=2.5cm, bottom=2.5cm}  % 设置页边距
\usepackage{graphicx}  % 引用图片
\usepackage{ctex}
\usepackage{fontspec}
\usepackage{setspace}

% 字体设置
\setmainfont{Times New Roman}
\setCJKmainfont{SimSun}
\setCJKsansfont{SimHei}

% 表格设置
\usepackage{makecell}
\newcommand{\addcell}[2][4]{\makecell{\zihao{#1}\textsf{#2}}}
\usepackage{titlesec}
\usepackage{booktabs}
\usepackage{tabularx}

% 设置图注、表注
\usepackage{caption}
\usepackage{bicaption}
\captionsetup{labelsep=quad, font={small, bf}, skip=2pt}
\DeclareCaptionOption{english}[]{
    \renewcommand\figurename{Fig.}
    \renewcommand\tablename{Table}
}
\captionsetup[bi-second]{english}

% 设置页眉
\usepackage{fancyhdr}
\pagestyle{fancy}
\fancypagestyle{preContent}{
    \fancyhead[L]{\zihao{-5} 物理化学实验}
    \fancyhead[C]{\zihao{-5} 实验三\ \ 液体饱和蒸气压的测定}
    \fancyhead[R]{\zihao{-5} 1800011716\ 王崇斌}
}
\pagestyle{preContent}

%	设置首页页眉页脚
\fancypagestyle{plain}{
	\fancyhead[L]{\zihao{-5} 物理化学实验}
	\fancyhead[C]{\zihao{-5} 实验三\ \ 液体饱和蒸气压的测定}
	\fancyhead[R]{\zihao{-5} 1800011716\ 王崇斌}
	\cfoot{}
}

% 设置标题格式
\titleformat*{\section}{\zihao{4}\sffamily}
\titleformat*{\subsection}{\zihao{-4}\sffamily}
\titleformat*{\subsubsection}{\zihao{-4}\sffamily}
\titlespacing*{\section}{0pt}{10pt}{10pt}
\titlespacing*{\subsection}{0pt}{10pt}{5pt}
\titlespacing*{\subsubsection}{0pt}{10pt}{5pt}

% 设置引用格式
\usepackage[super,round,comma,compress]{natbib}
\usepackage{hyperref}  % 使用hyperref包,可以提供文献引用到文件末尾


% 一些相关的包
\usepackage{amsmath}  % 数学公式
\usepackage{amssymb}  % 特殊字符
\usepackage[version=4]{mhchem}  % 用于输入化学式
\usepackage{braket}  % 用于输入Dirac符号
\usepackage{subfigure}  % 多张图片的排版

% 定义常用的命令
\def\d{\mathrm{d}}  % 正体的常用数学常数
\def\e{\mathrm{e}}
\def\i{\mathrm{i}}
\def\dps{\displaystyle}  % 
\newcommand{\mr}[1]{\mathrm{#1}}
\newcommand{\mb}[1]{\mathbf{#1}}
\newcommand{\dv}[2]{\frac{\d{#1}}{\d{#2}}}  % 定义导数、偏导数的简便记号
\newcommand{\pdv}[2]{\frac{\partial{#1}}{\partial{#2}}}
\def\degree{$^{\circ}$}  % 角度
\def\celsius{^{\circ}\mr{C}}  % 摄氏度

%正文
\begin{document}
    % 标题页
    \begin{titlepage}
    	% 页眉
    	\thispagestyle{plain}
        % 图片
        \begin{figure}[h]
            \centering
            \includegraphics{pku.png}
        \end{figure}
        \vspace{24pt}
        % 标题
        \centerline{\zihao{-0} \textsf{物理化学实验报告}}
        \vspace{40pt} % 空行
        \begin{center}
            \begin{tabular}{cp{14.1cm}}
                % 题目
                \addcell[2]{题目:\ } & \addcell[2]{液体饱和蒸气压的测定} \\
                \cline{2-2}
            \end{tabular}
        \end{center}
        \vspace{20pt} % 空行
        \begin{center}
            \doublespacing
            \begin{tabular}{cp{5cm}}
                % 姓名
                \addcell{姓\phantom{空格}名:\ } & \addcell{王崇斌} \\
                \cline{2-2}
                % 学号
                \addcell{学\phantom{空格}号:\ } & \addcell{1800011716}\\
                \cline{2-2}
                % 组别
                \addcell{组\phantom{空格}别:\ } & \addcell{19组} \\
                \cline{2-2}
                % 实验日期
                \addcell{实验日期:\ } & \addcell{2021.09.23}\\
                \cline{2-2}
                % 室温
                \addcell{室\phantom{空格}温:\ } & \addcell{197.26\ K}\\
                \cline{2-2}
                % 大气压强
                \addcell{大气压强:\ } & \addcell{101.18\ kPa
				\footnote{这个是动态法测定水的实验中测量的大气压,在静态法测定四氯化碳
				的实验中测得的气压为101.07Kpa}
				}\\
                \cline{2-2}
            \end{tabular}
            \begin{tabular*}{\textwidth}{c}
                \\ % 这是空行
                \\ % 这是空行
                \\ % 这是空行
                \\ % 这是空行
                \hline % 分割线
            \end{tabular*}
        \end{center}
        % 摘要
        \textsf{摘\ \ 要}\ \ 本实验使用静态法测定不同温度下\ce{CCl4}的饱和蒸气压;
		使用动态法测定水在不同压力下的沸点。利用
		Clausius-Clapeyron方程拟合实验结果,得到\ce{CCl4}和水的$\Delta_{vap}H_{m}$分别为
		$31.43 \pm 0.07\,\mr{kJ/mol}$和$41.20 \pm 0.09\,\mr{kJ/mol}$。
		在$p = 101.3\mr{kPa}$下的沸点分别为$76.03\celsius$和$99.73\celsius$;
		摩尔气化熵分别为$90.01\,\mr{J/(mol\,K)}$和$110.50\,\mr{J/(mol\,K)}$。
		由实验结果知,\ce{CCl4}对Trouton规则符合得较好,而水产生了明显的正偏差,
		其原因是氢键的存在使得液态水中形成了有序结构,降低了液态水的熵。
        \\
        \\
        % 关键字
        \textsf{关键词}\ \ Clapeyron方程;饱和蒸汽压;沸点
    \end{titlepage}
	\vbox{}        
    \section{实验部分}
    	\subsection{仪器和试剂}
    	\begin{enumerate}
			\item 试剂:\ce{CCl4},二次去离子水
			\item 仪器:数字式温度-压力测定仪,电加热器,循环水真空泵,冷凝水循环系统,真空缓冲罐,磁力搅拌器等
		\end{enumerate}
    	
		\subsection{实验步骤}
			首先在电脑上安装读取、记录数据的驱动程序,将数字式温度-压力测定仪与电脑连接,调试。
			\subsubsection{静态法测定\ce{CCl4}的饱和蒸汽压}
			按照讲义指导搭建实验装置。首先进行检漏:打开抽气阀门,装置减压至$50\mr{kPa}$左右,关闭抽气阀门,
			如果3min内气压的变化不超过$0.1\mr{kPa}$,说明气密性良好。\\
			开平衡阀通大气,加热至$80\celsius$左右使\ce{CCl4}沸腾,观察到剧烈冒泡;加热过程中,
			c管中的空气逐渐被压入b管排出。关闭加热,观察到沸腾停止,随后b管液面下降,bc两管中液面逐渐接近,
			待c管中无气泡且液面与b管液面水平时,按下“Hold”键,记下压强值和温度值,重复3次,如果度数接近,
			说明空气排尽。\\
			不断降低装置内压力,每次约降低$5\mr{kPa}$,随后测定液面水平时的压强值和温度值,
			直到压力降到约为$50\mr{kPa}$。

			\subsubsection{动态法测定水的饱和蒸汽压}
			在两口烧瓶中盛入约200 mL的去离子水。在各个玻璃接口处涂抹真空脂,按照讲义连接装置,
			使温度探头的前端与水面相切,同前述方法检查气密性。打开冷凝水,
			减压至示数为$50\mr{kPa}$左右,搅拌加热至沸腾,温度和压力稳定后记录数值。
			然后逐渐升高压力(每次约5 kPa),重复记录数据至与大气连通,大气压下的沸点测三次。

	\vbox{}  
	\section{数据与结果}
 		\subsection{实验原始数据记录及处理}
 		\subsubsection{静态法测定\ce{CCl4}的饱和蒸汽压}
		装置检漏的数据参见表\ref{static detection}。
		\begin{table}[h]
			\centering
			\zihao{5}
			\bicaption{装置检漏时的气压记录}{pressure document in leak detection}
			\begin{tabular}{ccc}
				\toprule
				时间 & P/kPa & T$\celsius$ \\
				\midrule
				12:55 & 49.75 & 22.75  \\
				12:57 & 49.81 & 22.74  \\
				12:58 & 49.84 & 22.74  \\
				\bottomrule
			\end{tabular}
			\label{static detection}
		\end{table}
		可以认为装置在实验过程中不漏气。随
 		\begin{figure}[h]
 			\centering
 			\includegraphics[width=0.5\textwidth]{pku.png}
 			\bicaption{中文图题}{Caption}
 		\end{figure}
 	
 		xxx如\textbf{表1}所示。
 		
 		  \begin{table}[h]
 			\centering
 			\zihao{5}
 			\bicaption{中文表题}{Caption}
 			\begin{tabular}{ccccc}
 				\toprule
 				一 & 二 & 三 & 四 & \thead[c] {$E_{trans}$ \\ / $ \rm kJ \cdot mol^{-1} $} \\
 				\midrule
 				1 & 1 & 1 & 3 & $-1.01316\times10^{6}$ \\
 				2 & 2 & 1 & 7 & \\
 				3 & 3 & 1 & 3 & \\
 				4 & 4 & 1 & 6 & \\
 				5 & 5 & 1 & 1 & \\
 				\bottomrule
 			\end{tabular}
 		\end{table}
 		\vbox{}
 		\subsubsection{动态法测定水的饱和蒸汽压}

 	
 		参考值\citealp{dean1992lange} 

 		\subsection{实验结果}
 		\subsubsection{静态法测定\ce{CCl4}的饱和蒸汽压}
 		\subsubsection{动态法测定水的饱和蒸汽压}
 		
 		

 	
	\vbox{}  	
 	\section{讨论与结论}
		\subsection{实验讨论}
 		\subsubsection{实验讨论1}
 	 	实验讨论1...
 	 	\subsubsection{实验讨论2}
 	 	实验讨论2...
 	 	\subsection{实验结论}
 	 	实验结论...

	\vbox{}  

	\bibliographystyle{achemso}
	\bibliography{cite}



\end{document}